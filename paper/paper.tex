%%
%% Beginning of file 'sample61.tex'
%%
%% Modified 2016 September
%%
%% This is a sample manuscript marked up using the
%% AASTeX v6.1 LaTeX 2e macros.
%%
%% AASTeX is now based on Alexey Vikhlinin's emulateapj.cls 
%% (Copyright 2000-2015).  See the classfile for details.

%% AASTeX requires revtex4-1.cls (http://publish.aps.org/revtex4/) and
%% other external packages (latexsym, graphicx, amssymb, longtable, and epsf).
%% All of these external packages should already be present in the modern TeX 
%% distributions.  If not they can also be obtained at www.ctan.org.

%% The first piece of markup in an AASTeX v6.x document is the \documentclass
%% command. LaTeX will ignore any data that comes before this command. The 
%% documentclass can take an optional argument to modify the output style.
%% The command below calls the preprint style  which will produce a tightly 
%% typeset, one-column, single-spaced document.  It is the default and thus
%% does not need to be explicitly stated.
%%
%%
%% using aastex version 6.1
\documentclass[modern]{aastex61}

%% The default is a single spaced, 10 point font, single spaced article.
%% There are 5 other style options available via an optional argument. They
%% can be envoked like this:
%%
%% \documentclass[argument]{aastex61}
%% 
%% where the arguement options are:
%%
%%  twocolumn   : two text columns, 10 point font, single spaced article.
%%                This is the most compact and represent the final published
%%                derived PDF copy of the accepted manuscript from the publisher
%%  manuscript  : one text column, 12 point font, double spaced article.
%%  preprint    : one text column, 12 point font, single spaced article.  
%%  preprint2   : two text columns, 12 point font, single spaced article.
%%  modern      : a stylish, single text column, 12 point font, article with
%% 		  wider left and right margins. This uses the Daniel
%% 		  Foreman-Mackey and David Hogg design.
%%
%% Note that you can submit to the AAS Journals in any of these 6 styles.
%%
%% There are other optional arguments one can envoke to allow other stylistic
%% actions. The available options are:
%%
%%  astrosymb    : Loads Astrosymb font and define \astrocommands. 
%%  tighten      : Makes baselineskip slightly smaller, only works with 
%%                 the twocolumn substyle.
%%  times        : uses times font instead of the default
%%  linenumbers  : turn on lineno package.
%%  trackchanges : required to see the revision mark up and print its output
%%  longauthor   : Do not use the more compressed footnote style (default) for 
%%                 the author/collaboration/affiliations. Instead print all
%%                 affiliation information after each name. Creates a much
%%                 long author list but may be desirable for short author papers
%%
%% these can be used in any combination, e.g.
%%
%% \documentclass[twocolumn,linenumbers,trackchanges]{aastex61}

%% AASTeX v6.* now includes \hyperref support. While we have built in specific
%% defaults into the classfile you can manually override them with the
%% \hypersetup command. For example,
%%
%%\hypersetup{linkcolor=red,citecolor=green,filecolor=cyan,urlcolor=magenta}
%%
%% will change the color of the internal links to red, the links to the
%% bibliography to green, the file links to cyan, and the external links to
%% magenta. Additional information on \hyperref options can be found here:
%% https://www.tug.org/applications/hyperref/manual.html#x1-40003

%% If you want to create your own macros, you can do so
%% using \newcommand. Your macros should appear before
%% the \begin{document} command.
%%
\newcommand{\vdag}{(v)^\dagger}
\newcommand\aastex{AAS\TeX}
\newcommand\latex{La\TeX}

\newcommand{\acronym}[1]{{\small{#1}}}
\newcommand{\project}[1]{\textsl{#1}}
\newcommand{\kepler}{\project{Kepler}}
\newcommand{\tess}{\project{\acronym{TESS}}}
\newcommand{\gaia}{\project{Gaia}}
\newcommand{\lsst}{\project{\acronym{LSST}}}
\newcommand{\bsa}{\project{\acronym{BSA}}}

\newcommand{\normal}{\ensuremath{{\rm N}}}

\newcommand{\prob}{\ensuremath{\rm Pr}}
\newcommand{\pars}{\ensuremath{\theta}}
\newcommand{\parsnl}{\ensuremath{\theta_{\rm nl}}}
\newcommand{\parsl}{\ensuremath{\theta_{\rm lin}}}

\newcommand{\nt}{\ensuremath{n_{t}}}
\newcommand{\nlin}{\ensuremath{n_{\rm lin}}}

\newcommand{\nukl}{\ensuremath{\nu_{nk\ell}}}
\newcommand{\nukzero}{\ensuremath{\nu_{nk0}}}
\newcommand{\nukone}{\ensuremath{\nu_{nk1}}}
\newcommand{\nuzero}{\ensuremath{\nu_{n00}}}
\newcommand{\dnu}{\ensuremath{\Delta\nu}}
\newcommand{\numax}{\ensuremath{\nu_{\rm max}}}
\newcommand{\numod}{\ensuremath{{\rm bell}}}
\newcommand{\amps}{\ensuremath{A_{nk\ell}}}
\newcommand{\ampc}{\ensuremath{B_{nk\ell}}}
\newcommand{\eps}{\ensuremath{\kappa_{01}}}
\newcommand{\rl}{\ensuremath{r_{01}}}

%% Reintroduced the \received and \accepted commands from AASTeX v5.2
\received{TBD}
\revised{NA}
\accepted{NA}
%% Command to document which AAS Journal the manuscript was submitted to.
%% Adds "Submitted to " the arguement.
\submitjournal{ApJ}

%% Mark up commands to limit the number of authors on the front page.
%% Note that in AASTeX v6.1 a \collaboration call (see below) counts as
%% an author in this case.
%
%\AuthorCollaborationLimit=3
%
%% Will only show Schwarz, Muench and "the AAS Journals Data Scientist 
%% collaboration" on the front page of this example manuscript.
%%
%% Note that all of the author will be shown in the published article.
%% This feature is meant to be used prior to acceptance to make the
%% front end of a long author article more manageable. Please do not use
%% this functionality for manuscripts with less than 20 authors. Conversely,
%% please do use this when the number of authors exceeds 40.
%%
%% Use \allauthors at the manuscript end to show the full author list.
%% This command should only be used with \AuthorCollaborationLimit is used.

%% The following command can be used to set the latex table counters.  It
%% is needed in this document because it uses a mix of latex tabular and
%% AASTeX deluxetables.  In general it should not be needed.
%\setcounter{table}{1}

%%%%%%%%%%%%%%%%%%%%%%%%%%%%%%%%%%%%%%%%%%%%%%%%%%%%%%%%%%%%%%%%%%%%%%%%%%%%%%%%
%%
%% The following section outlines numerous optional output that
%% can be displayed in the front matter or as running meta-data.
%%
%% If you wish, you may supply running head information, although
%% this information may be modified by the editorial offices.
\shorttitle{\aastex\ sample article}
\shortauthors{Feeney et al.}
%%
%% You can add a light gray and diagonal water-mark to the first page 
%% with this command:
% \watermark{text}
%% where "text", e.g. DRAFT, is the text to appear.  If the text is 
%% long you can control the water-mark size with:
%  \setwatermarkfontsize{dimension}
%% where dimension is any recognized LaTeX dimension, e.g. pt, in, etc.
%%
%%%%%%%%%%%%%%%%%%%%%%%%%%%%%%%%%%%%%%%%%%%%%%%%%%%%%%%%%%%%%%%%%%%%%%%%%%%%%%%%

%% This is the end of the preamble.  Indicate the beginning of the
%% manuscript itself with \begin{document}.

\begin{document}\sloppy\sloppypar\raggedbottom\frenchspacing % trust the Hogg

\title{Asteroseismology {\it sans Fourier}: soon to be crushed by DFM}

%% LaTeX will automatically break titles if they run longer than
%% one line. However, you may use \\ to force a line break if
%% you desire. In v6.1 you can include a footnote in the title.

%% A significant change from earlier AASTEX versions is in the structure for 
%% calling author and affilations. The change was necessary to implement 
%% autoindexing of affilations which prior was a manual process that could 
%% easily be tedious in large author manuscripts.
%%
%% The \author command is the same as before except it now takes an optional
%% arguement which is the 16 digit ORCID. The syntax is:
%% \author[xxxx-xxxx-xxxx-xxxx]{Author Name}
%%
%% This will hyperlink the author name to the author's ORCID page. Note that
%% during compilation, LaTeX will do some limited checking of the format of
%% the ID to make sure it is valid.
%%
%% Use \affiliation for affiliation information. The old \affil is now aliased
%% to \affiliation. AASTeX v6.1 will automatically index these in the header.
%% When a duplicate is found its index will be the same as its previous entry.
%%
%% Note that \altaffilmark and \altaffiltext have been removed and thus 
%% can not be used to document secondary affiliations. If they are used latex
%% will issue a specific error message and quit. Please use multiple 
%% \affiliation calls for to document more than one affiliation.
%%
%% The new \altaffiliation can be used to indicate some secondary information
%% such as fellowships. This command produces a non-numeric footnote that is
%% set away from the numeric \affiliation footnotes.  NOTE that if an
%% \altaffiliation command is used it must come BEFORE the \affiliation call,
%% right after the \author command, in order to place the footnotes in
%% the proper location.
%%
%% Use \email to set provide email addresses. Each \email will appear on its
%% own line so you can put multiple email address in one \email call. A new
%% \correspondingauthor command is available in V6.1 to identify the
%% corresponding author of the manuscript. It is the author's responsibility
%% to make sure this name is also in the author list.
%%
%% While authors can be grouped inside the same \author and \affiliation
%% commands it is better to have a single author for each. This allows for
%% one to exploit all the new benefits and should make book-keeping easier.
%%
%% If done correctly the peer review system will be able to
%% automatically put the author and affiliation information from the manuscript
%% and save the corresponding author the trouble of entering it by hand.

\correspondingauthor{Stephen M. Feeney}
\email{sfeeney@flatironinstitute.org}

\author[0000-0003-2268-2519]{Stephen M. Feeney}
\affil{Center for Computational Astrophysics, Flatiron Institute, 162 Fifth Ave, New York, NY 10010, USA}

\author[0000-0003-2866-9403]{David W. Hogg}
\affiliation{Center for Computational Astrophysics, Flatiron Institute, 162 Fifth Ave, New York, NY 10010, USA}
\affiliation{Center for Cosmology and Particle Physics, Department of Physics, New York University, 726 Broadway, room 1005, New York, NY 10003, USA }
\affiliation{Center for Data Science, New York University, 60 Fifth Ave, New York, NY 10011, USA }
\affiliation{Max-Planck-Institut f\"ur Astronomie, K\"onigstuhl 17, D-69117 Heidelberg, Germany }

%% Note that the \and command from previous versions of AASTeX is now
%% depreciated in this version as it is no longer necessary. AASTeX 
%% automatically takes care of all commas and "and"s between authors names.

%% AASTeX 6.1 has the new \collaboration and \nocollaboration commands to
%% provide the collaboration status of a group of authors. These commands 
%% can be used either before or after the list of corresponding authors. The
%% argument for \collaboration is the collaboration identifier. Authors are
%% encouraged to surround collaboration identifiers with ()s. The 
%% \nocollaboration command takes no argument and exists to indicate that
%% the nearby authors are not part of surrounding collaborations.

%% Mark off the abstract in the ``abstract'' environment. 
\begin{abstract}\noindent % trust the Hogg
Stellar oscillations are stochastically driven high-quality (long-coherence-time) modes.
If a star is observed through an observing campaign that is shorter
than the coherence time of a mode, that mode will effectively appear
coherent within that time interval.
Here we exploit this to create an inexpensive probabilistic model for
asteroseismology data called \bsa.
The huge advantage of \bsa\ over traditional asteroseismology
approaches is that it does not require the performance of a fourier
transform or anything like it; this permits the measurement of
asteroseismic parameters in data sets that are irregularly or sparsely
sampled, or in data sets where the critical modes are near the edges
of the available frequency space (near the inverse exposure time at
the high-frequency end or near the inverse of the duration of the
campaign at the low-freqency end).
It also permits the construction of a tractable and justifiable
likelihood function, even when the noise processes are non-trivial.
We produce a parameterized description of an asteroseismic frequency
spectrum and use it along with this likelihood function to produce
posterior estimates of asteroseismic parameters for stars in the
\kepler\ data.
We sub-sample the \kepler\ light curves to (short duration)
\tess-like and (sparsely sampled) \gaia-like data sets and show that
even in these much smaller data sets, we ought to be able to estimate
asteroseismic parameters for many stars, and (because we are just
turning the Bayesian crank) without human intervention.
Much of what is done here could be done with Gaussian Processes; the
advantages of \bsa\ are that it is conceptually simple, and that it
has better scaling for some classes of problems.
\end{abstract}

%% Keywords should appear after the \end{abstract} command. 
%% See the online documentation for the full list of available subject
%% keywords and the rules for their use.
\keywords{editorials, notices --- miscellaneous --- catalogs ---
  surveys --- methods of limited utility}

%% From the front matter, we move on to the body of the paper.
%% Sections are demarcated by \section and \subsection, respectively.
%% Observe the use of the LaTeX \label
%% command after the \subsection to give a symbolic KEY to the
%% subsection for cross-referencing in a \ref command.
%% You can use LaTeX's \ref and \label commands to keep track of
%% cross-references to sections, equations, tables, and figures.
%% That way, if you change the order of any elements, LaTeX will
%% automatically renumber them.

%% We recommend that authors also use the natbib \citep
%% and \citet commands to identify citations.  The citations are
%% tied to the reference list via symbolic KEYs. The KEY corresponds
%% to the KEY in the \bibitem in the reference list below. 

\section{Introduction}\label{sec:intro}

Asteroseismology is performed by taking continuous, excellent data.

Parameter estimation is done by looking at the FT of the time-domain data.

The FT is much cleaner when the data is regularly spaced, close to
homoskedastic, and taken identically.

For example, the FT of the data is the multiply of the FT of the star
with the FT of the exposure-time window, so the whole project breaks
if the exposure time is varied.

For another example, there is no traditional way to do
asteroseismology with sparse-in-time data, like that which will be
produced by \lsst\ and \gaia.

For yet another: Time-localized outlers are death in the Fourier domain.

But in asteroseismology we are only trying to learn a very small number
of scalars from the light curve. Do we really need to beat the light curve
against every possible period?

Finally: Stars are damned coherent. Red giants, in particular, show little
or no sign of decoherence over the entire duration of the \kepler\ Mission.
That should make life better; how?

Here we capitalize on the near-coherence of the modes over reasonable
time periods (and the magic of Gaussians) to perform asteroseismic
parameter estimation in the time domain, without any use of the
Fourier transform.

\section{Method}\label{sec:method}

Consider a set of $2K+1$ modes, indexed by $k \in \mathbb{Z}: -K \le k \le K$. Assuming that the $\ell = 0, 1$ modes are present, we expect a comb of $4K+2$ peaks in the power spectrum. The $\ell = 0$ modes are located at
\begin{equation}
\nukzero = \nuzero \, (1 + k \, \Delta), 
\end{equation}
with the $\ell = 1$ modes shifted to
\begin{equation}
\nukone = \nuzero \, (1 + [k + \eps] \, \Delta), 
\end{equation}
with $\eps \simeq 1/2$. Each of these modes corresponds to two sinusoids
\begin{equation}
m_{k\ell}(t) = \amps \, \sin(2\pi \, \nukl \, t) + \ampc \, \cos(2\pi \, \nukl \, t)
\end{equation}
into the stellar lightcurve, yielding $\nlin = 8K+4$ unknown amplitude parameters. The total model lightcurve (i.e., summed over $k$ and $\ell$) can be represented by
\begin{equation}
m(t) = D(t, \nukl) \, \parsl
\end{equation}
where $D$ is an $\nt \, \times \, \nlin$ design matrix and
\begin{equation}
\parsl = \{ A_{n,-K,0}, B_{n,-K,0}, A_{n,-K,1}, B_{n,-K,1}, A_{n,-K+1,0}, B_{n,-K+1,0}, \ldots, A_{n,K,1}, B_{n,K,1} \}
\end{equation}
is a vector of amplitudes. The amplitudes are Gaussian-distributed with a frequency- and $\ell$-dependent variance; for example, 
\begin{equation}
\amps \sim \normal \left(\amps \, ; \, 0, \rl^\ell \, \sigma^2(\nukl) \right),
\end{equation}
where $\rl$, the ratio of the variances of the $\ell = 0$ and $\ell = 1$ modes with equal $k$, is approximately 1/2. The variance, $\sigma^2(\nukl)$, of the amplitudes is modulated by a bell-shaped envelope, yielding
\begin{equation}
\sigma^2(\nu) = \numod(\nu \, ; \, \numax, W, H),
\end{equation}
where
\begin{equation}
\numod(\nu \, ; \, \numax, W, H) = H \, \exp \left( -\frac{[\nu - \numax]^2}{2 \, W^2} \right).
\end{equation}
This model is therefore completely described by the parameter vector $\pars = \{ \amps, \ampc, \nuzero, \Delta, \eps, \rl, \numax, W, H \}$, depending linearly on the amplitudes, $\parsl = \{\amps, \ampc\}$, and non-linearly on the remainder, $\parsnl$.

In the following, we assume $K=16$, yielding a total of 132 linear parameters and seven non-linear parameters. Our aim is to recover the marginal posterior of $\numax$ and $\dnu = \nuzero \, \Delta$, which can be estimated by sampling from the full joint posterior distribution
\begin{equation}
\prob(\pars \, | \, d(t)) \propto \prob(\pars) \, \prob(d(t) \, | \, \pars).
\end{equation}
Na\"ively this involves the daunting task of sampling from a highly multimodal distribution in 140 dimensions. We can, however, greatly reduce the dimensionality by exploiting the linear dependence of the model on the amplitudes. Assuming Gaussian noise with time-domain covariance matrix $C$ yields a Gaussian likelihood function
\begin{equation}
\prob(d(t) \, | \, \pars) = \normal \left(d(t) \, ; \, m(t, \pars), C \right),
\end{equation}
which, coupled with the assumption of Gaussian-distributed amplitudes means we can pre-marginalize over them analytically, yielding
\begin{eqnarray}
\prob(d(t) \, | \, \parsnl) & = & \int d\parsl \, \prob(\parsl) \, \prob(d(t) \, | \, \pars) \nonumber \\
& = & \normal \left(d(t) \, ; \, 0, V(\parsnl) \right).
\end{eqnarray}
Here, the modified covariance matrix $V = C + D \, \Lambda \, D^T$, where $\Lambda$ is a diagonal $\nlin^2$ matrix with non-zero entries 
\begin{equation}
\parsl = \{ \sigma^2(\nu_{0,-K,0}), \sigma^2(\nu_{0,-K,0}), \sigma^2(\nu_{0,-K,1}), \sigma^2(\nu_{0,-K,1}), \ldots, \sigma^2(\nu_{0,K,1}), \sigma^2(\nu_{0,K,1}) \}.
\end{equation}

The slowest steps in evaluating this likelihood are inverting and finding the determinant of the $\nt^2$ covariance matrix $V$; however, provided $\nlin < \nt$ we can increase the speed of this step (potentially significantly: by a factor of $(\nt/\nlin)^3$) by performing low-rank updates of the data covariance: 
\begin{eqnarray}
V^{-1} & = & C^{-1} - C^{-1} \, D \, \left(I + \Lambda \, D^T \, C \, D \right)^{-1} \Lambda \, D^T \, C^{-1} \\
||V|| & = & ||I + \Lambda \, D^T \, C \, D|| \, ||C||.
\end{eqnarray}
Using these forms of the two matrix lemmas is important, as the bell function modulating the variances of the linear amplitudes allows for singular $\Lambda$ matrices, particularly when $W$ is small.

\end{document}

% End of file `sample61.tex'.
